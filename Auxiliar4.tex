\documentclass[11pt, spanish]{article}
\pagestyle{empty}

%acentos de la forma � en vez de \'a
\usepackage[spanish]{babel}
\usepackage[utf8]{inputenc}

%Enumeración con columnas que se usa con \begin{multicols}{2} \begin{enumerate}
\usepackage{multicol}
\usepackage{multirow}


%Para poder usar begin comment
\usepackage{verbatim}

%Teoremas
\usepackage{amsthm}
\theoremstyle{plain}
\newtheorem{teo}{Teorema}


%letras para enumerar
\makeatletter\renewcommand\theenumi{\@alph\c@enumi}\makeatother\renewcommand\labelenumi{\theenumi)}

%m�rgenes
\usepackage[left=2cm,top=2cm,right=2cm]{geometry}

\usepackage{fancyhdr}
\pagestyle{fancy}
\usepackage{enumerate}

%conjuntos N, Q, R, Z, C
\usepackage{dsfont}
\newcommand{\N}{\mathds{N}}
\newcommand{\Q}{\mathds{Q}}
\newcommand{\R}{\mathds{R}}
\newcommand{\Z}{\mathds{Z}}
\newcommand{\C}{\mathds{C}}
\newcommand{\M}{\mathcal{M}}
\newcommand{\Pol}{\mathcal{P}}

%Transformada de Laplace
\renewcommand{\L}{\mathcal{L}}

%Probabilidades
\newcommand{\PP}{\mathbb{P}}
\newcommand{\E}{\mathbb{E}}
\newcommand{\B}{\mathcal{B}}
\newcommand{\Var}{\mathbb{V}\text{ar}}

%l�gica
\newcommand{\ssi}{\Leftrightarrow}

%funciones
\newcommand{\function}[5]{  \begin{array}{rl}
                                #1: #2 &\longrightarrow #3 \\
                                #4 & \longmapsto #1\left(#4\right)= #5
                            \end{array} }

\newcommand{\funcion}[3]{#1: #2 \longrightarrow #3 }
\newcommand{\parteentera}[1]{[#1]}

%%%operadores matematicos
\usepackage{mathtools}
\DeclarePairedDelimiter\abs{\lvert}{\rvert}%
\DeclarePairedDelimiter\norm{\lVert}{\rVert}%
\newcommand{\tr}{\text{tr}}


% Swap the definition of \abs* and \norm*, so that \abs
% and \norm resizes the size of the brackets, and the 
% starred version does not.
\makeatletter
\let\oldabs\abs
\def\abs{\@ifstar{\oldabs}{\oldabs*}}
%
\let\oldnorm\norm
\def\norm{\@ifstar{\oldnorm}{\oldnorm*}}
\makeatother
% % % % % % % % % %
%\providecommand{\abs}[1]{\lvert#1 \rvert}
%\providecommand{\norm}[1]{\lVert#1 \rVert}
%\providecommand{\pin}[2]{\left< #1,#2 \right>} %producto interno

\providecommand{\dpartial}[2]{\frac{\partial #1}{\partial #2}} %derivada parcial

\usepackage{amssymb}
\usepackage{amsmath, amsthm, amsfonts}

%Im�genes
\usepackage{graphicx}
\usepackage{float}
\begin{document}
\fancyhead[L]{Facultad de Ciencias Físicas y Matemáticas}
\fancyhead[R]{Universidad de Chile}

\begin{flushleft}
  \textbf{IN6228-1 Teoría de Juegos 2018}
  \\\textbf{Profesor:} José Correa.
  \\\textbf{Auxiliares:} Carlos Bonet y Andrés Cristi.
\end{flushleft}


\begin{center}
  \large{\textbf{Clase Auxiliar 4\\ 23 de Abril de 2018}}
\end{center}

%\begin{flushleft}



\begin{itemize}
\item[\textbf{P1.}] 
	\begin{enumerate}
		\item Recuerde el juego de Cournot de $n$ firmas, donde cada una
			escoge una cantidad $s_i$ a producir y sus pagos son
			\begin{align*}
				p_i(s)= s_i \left( a- b\sum_{j=1}^n s_j \right) -c_i s_i
			\end{align*}
			con $c_i$ el costo de producción. Encuentre un potencial y con \'el 
			encuentre un EN en el caso $c_i=c, \forall i$.
		\item Muestre que el siguiente juego no admite potencial
			\[
				\left( \begin{matrix} 2,2 & 0,3 \\ 3,0 & 1,2 \end{matrix}\right)
			\]
		\item Suponga que el jugador de las filas juega primero y desp\'es el de las columnas.
			Cuál es el EPS?
		\item En el juego anterior, suponga que un mediador propone con probabilidad
			$1/2$ TL y con probabilidad $1/2$ BR. Es esto un equilibrio
			correlacionado? Muestre que existe un único equilibrio correlacionado.

	\end{enumerate}

\item[\textbf{P2.}] Un conjunto de $m$ máquinas id\'enticas son compartidas por $n$
	agentes, cada uno de los cuales quiere procesar su tarea. La tarea del
	agente $j$ demora un tiempo $p_j$ en ser procesada y quiere que est\'e
	terminada lo antes posible. Para ello, cada agente elige una de las máquinas
	y las máquinas procesan los trabajos que recibieron según la regla de Smith,
	es decir, en orden creciente de tamaño. Probaremos que el PoA de este juego
	es finito, al tomar como costo social $\sum_j c_j$, con $c_j$ el tiempo
	de completación de la tarea $j$.
	\begin{enumerate}
		\item Explique por qu\'e se puede suponer que todos los $p_j$ son
			distintos.
		\item Pruebe que siempre existe EN en estrategias puras.
		\item Sea $J_i$ el conjunto de tareas que recibe la máquina $i$.
			Pruebe que en el equilibrio la tarea $j$ es procesada
			en una máquina $i$ que minimiza 
			\begin{align*}
				\sum_{k\in J_i: p_k\leq p_j} p_k.
			\end{align*}
		\item Imagine que tiene una máquina que es $m$ veces más rápida
			que las máquinas originales y llame OPTF al valor 
			óptimo en ella. Pruebe que el costo social de un equilibrio
			EQ es a lo más OPTF $+ \sum_j p_j$.
		\item Llamemos OPT al costo social de la asignación óptima en 
			las $m$ máquinas. Pruebe que OPTF $\leq$ OPT.
		\item Concluya que el PoA es a lo más 2.
	\end{enumerate}


	

\end{itemize}
\end{document} 
