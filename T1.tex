\documentclass[11pt, spanish]{article}
\pagestyle{empty}

%acentos de la forma � en vez de \'a
\usepackage[spanish]{babel}
\usepackage[utf8]{inputenc}

%Enumeración con columnas que se usa con \begin{multicols}{2} \begin{enumerate}
\usepackage{multicol}
\usepackage{multirow}


%Para poder usar begin comment
\usepackage{verbatim}

%Teoremas
\usepackage{amsthm}
\theoremstyle{plain}
\newtheorem{teo}{Teorema}


%letras para enumerar
\makeatletter\renewcommand\theenumi{\@alph\c@enumi}\makeatother\renewcommand\labelenumi{\theenumi)}

%m�rgenes
\usepackage[left=2cm, top=2cm, right=2cm, headheight=13.6pt]{geometry}

\usepackage{fancyhdr}
\pagestyle{fancy}
\usepackage{enumerate}

%conjuntos N, Q, R, Z, C
\usepackage{dsfont}
\newcommand{\N}{\mathds{N}}
\newcommand{\Q}{\mathds{Q}}
\newcommand{\R}{\mathds{R}}
\newcommand{\Z}{\mathds{Z}}
\newcommand{\C}{\mathds{C}}
\newcommand{\M}{\mathcal{M}}
\newcommand{\Pol}{\mathcal{P}}

%Transformada de Laplace
\renewcommand{\L}{\mathcal{L}}

%Probabilidades
\newcommand{\PP}{\mathbb{P}}
\newcommand{\E}{\mathbb{E}}
\newcommand{\B}{\mathcal{B}}
\newcommand{\Var}{\mathbb{V}\text{ar}}

%l�gica
\newcommand{\ssi}{\Leftrightarrow}

%funciones
\newcommand{\function}[5]{  \begin{array}{rl}
                                #1: #2 &\longrightarrow #3 \\
                                #4 & \longmapsto #1\left(#4\right)= #5
                            \end{array} }

\newcommand{\funcion}[3]{#1: #2 \longrightarrow #3 }
\newcommand{\parteentera}[1]{[#1]}

%%%operadores matematicos
\usepackage{mathtools}
\DeclarePairedDelimiter\abs{\lvert}{\rvert}%
\DeclarePairedDelimiter\norm{\lVert}{\rVert}%
\newcommand{\tr}{\text{tr}}


% Swap the definition of \abs* and \norm*, so that \abs
% and \norm resizes the size of the brackets, and the 
% starred version does not.
\makeatletter
\let\oldabs\abs
\def\abs{\@ifstar{\oldabs}{\oldabs*}}
%
\let\oldnorm\norm
\def\norm{\@ifstar{\oldnorm}{\oldnorm*}}
\makeatother
% % % % % % % % % %
%\providecommand{\abs}[1]{\lvert#1 \rvert}
%\providecommand{\norm}[1]{\lVert#1 \rVert}
%\providecommand{\pin}[2]{\left< #1,#2 \right>} %producto interno

\providecommand{\dpartial}[2]{\frac{\partial #1}{\partial #2}} %derivada parcial

\usepackage{amssymb}
\usepackage{amsmath, amsthm, amsfonts}

%Im�genes
\usepackage{graphicx}
\usepackage{float}
\begin{document}
\fancyhead[L]{Facultad de Ciencias Físicas y Matemáticas}
\fancyhead[R]{Universidad de Chile}

\begin{flushleft}
  \textbf{IN6228 - Teor\'ia de Juegos}
  \\\textbf{Profesor:} Jos\'e Correa.
  \\\textbf{Auxiliares:} Carlos Bonet, Andr\'es Cristi.
\end{flushleft}


\begin{center}
  \Large{\textbf{Tarea 1}}
\end{center}

%\begin{flushleft}



\begin{itemize}
  \item[\textbf{P1.}] Los jugadores 1 y 2 participan en un juego de suma-cero con matrices de pago $A^{1,2}$ y $A^{2,1}$; al mismo tiempo los jugadores 1 y 3 desarrollan otro juego de suma-cero con matrices de pago $A^{1,3}$ y $A^{3,1}$. En vez de ocupar diferentes estrategias en ambos juegos, el jugador 1 debe ocupar la misma estrategia en los dos juegos.
  \begin{enumerate}
    \item Muestre que existe un equilibrio de nash en este juego.
    \item De un algoritmo que permita calcular el equilibrio de nash. Comente sobre la eficiencia de su algoritmo propuesto.
  \end{enumerate}

  \item[\textbf{P2.}] Encuentre todos los equilibrios del juego matricial usando el algoritmo de Lemke-Howson:

     \begin{center}
      \begin{tabular}{|c|c|c|c|}
          \hline
            &L & C & R\\
          \hline\hline
          T&3,4& 5,3 & 2,3\\
          M& 2,5& 3,9& 4,6\\
          B& 3,1 & 2,5& 7,4\\
          \hline
      \end{tabular}
      \end{center}

  \item[\textbf{P3.}] Una jerarquizada manada de $n$ leones encuentran una presa. Si el le\'on 1 decide no comerse a la presa, \'esta escapa y el juego se acaba. Pero si el le\'on 1 se come a la presa, entonces \'el se pone gordo y lento, con esto el le\'on 2 se lo puede comer. Si el le\'on 2 no se come al le\'on 1 entonces el juego termina; sino el le\'on 3 puede comerse al 2 y as\'i sucesivamente se realiza el juego. Suponga que las preferencias de todos los leones son tales que prefieren comer antes de estar sin comida y no comer antes de ser asesinados. Modele el juego de manera extensiva y encuentre todos los EPS del juego.

  \item[\textbf{P4.}] 
    Dos rutas $A$ y $B$ conectan un cierto par 
  origen-destino, con tiempos de viaje que dependen del número de
  usuarios en cada ruta,

  \begin{align*}
    c_A(n_A) & = 1.2n_A - 0.4\\
    c_B(n_B) & = 0.8n_B + 0.4
  \end{align*}

  Consideremos dos jugadores los cuales deben escoger una ruta, de
  modo que sus conjuntos de estrategias son $S_1=S_2=\{A,B\}$.
  \begin{enumerate}
    \item Explicite la matriz de pagos del juego y determine todos
      los equilibrios de Nash.
    \item Encuentre el poliedro $P$ de los equilibrios correlacionados
      y verifique que todos los equilibrios encontrados en la parte
      anterior están en $P$.
    \item Si usted fuera árbitro de este juego, qu\'e equilibrio
      correlacionado escogería y justifique el porqu\'e.
    \item Consideremos un caso similar al anterior pero con $n$
      jugadores, los cuales escogen una ruta $r\in R=\{1,\dots,m\}$.
      El costo de la ruta $r$ es una función creciente $c_r(n_r)$ de
      la cantidad total $n_r$ de usuarios que eligen dicha ruta. Sea
      $s=(s_1,\dots,s_n)\in R^n$ un perfil de estrategias y definamos
      \begin{align*}
	U(s)= \sum_{r=1}^m \sum_{u=1}^{n_r} c_r(u)
      \end{align*}
      donde $n_r=|\{ s_i=r\}|$ es el número de usuarios que escogen
      la ruta $r$. Sea $s^*$ una solución óptima de $\min_{s\in R^n}
      U(s)$. Pruebe que $s^*$ es un equilibrio de Nash en estrategias
      puras.
  \end{enumerate}

\item[\textbf{P5.}]
  Un juego $\langle I, (S_i)_{i\in I}, (u_i)_{i\in I} \rangle$ se
  dice sim\'etrico si $S_i =S$ para todo $i\in I$ y $u_i(s_i,s_{-i})=
  u_j(s_j,s_{-j})$ para todo $i,j\in I$ cuando $s_i=s_j$ y $s_{-i}=s_{-j}$.
  A su vez, diremos que una estrategia $\sigma\in (\Delta S)^{I}$ es
  sim\'etrica cuando $\sigma_i= \sigma_j$ para todo par $i,j\in I$.

  \begin{enumerate}
    \item Para un juego sim\'etrico ¿Se tiene existencia de EN en
      estrategias puras? ¿Todo juego sim\'etrico admite un potencial?
      Demuestre o entregue contraejemplos.
    \item Para el caso de un juego sim\'etrico con $|S|\leq 2$,
      encuentre un EN puro mediante un algoritmo polinomial. Para
      este caso ¿se tiene existencia de EN puro y sim\'etrico?
      Demuestre o entregue un contraejemplo.
    \item Ahora probaremos la existencia de un EN sim\'etrico en
      el caso general. Para ello, considere para $i\in I, s\in S$ y
      $\sigma \in \Delta S$ las funciones
      \begin{align*}
	g_s(\sigma) & =  \max \big\{0, u_i(s,\sigma_{-i} -  \big\}, \text{ con } \sigma_{-i}=(\sigma,\dots,\sigma)\\
	y_s(\sigma) & = \frac{\sigma_s+g_s(\sigma)}{1+ \sum_{t\in S} g_t(\sigma)}
      \end{align*}
      Estudie los puntos fijos de $T(\sigma)= \big( y_s(\sigma)\big)_{s\in S}$ y concluya la existencia de
      un EN sim\'etrico.
  \end{enumerate}
  

\end{itemize}

\end{document} 
