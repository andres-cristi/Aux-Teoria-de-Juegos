\documentclass[11pt, spanish]{article}
\pagestyle{empty}

%acentos de la forma � en vez de \'a
\usepackage[spanish]{babel}
\usepackage[utf8]{inputenc}

%Enumeración con columnas que se usa con \begin{multicols}{2} \begin{enumerate}
\usepackage{multicol}
\usepackage{multirow}


%Para poder usar begin comment
\usepackage{verbatim}

%Teoremas
\usepackage{amsthm}
\theoremstyle{plain}
\newtheorem{teo}{Teorema}


%letras para enumerar
\makeatletter\renewcommand\theenumi{\@alph\c@enumi}\makeatother\renewcommand\labelenumi{\theenumi)}

%m�rgenes
\usepackage[left=2cm, top=2cm, right=2cm, headheight=13.6pt]{geometry}

\usepackage{fancyhdr}
\pagestyle{fancy}
\usepackage{enumerate}

%conjuntos N, Q, R, Z, C
\usepackage{dsfont}
\newcommand{\N}{\mathds{N}}
\newcommand{\Q}{\mathds{Q}}
\newcommand{\R}{\mathds{R}}
\newcommand{\Z}{\mathds{Z}}
\newcommand{\C}{\mathds{C}}
\newcommand{\M}{\mathcal{M}}
\newcommand{\Pol}{\mathcal{P}}

%Transformada de Laplace
\renewcommand{\L}{\mathcal{L}}

%Probabilidades
\newcommand{\PP}{\mathbb{P}}
\newcommand{\E}{\mathbb{E}}
\newcommand{\B}{\mathcal{B}}
\newcommand{\Var}{\mathbb{V}\text{ar}}

%l�gica
\newcommand{\ssi}{\Leftrightarrow}

%funciones
\newcommand{\function}[5]{  \begin{array}{rl}
                                #1: #2 &\longrightarrow #3 \\
                                #4 & \longmapsto #1\left(#4\right)= #5
                            \end{array} }

\newcommand{\funcion}[3]{#1: #2 \longrightarrow #3 }
\newcommand{\parteentera}[1]{[#1]}

%%%operadores matematicos
\usepackage{mathtools}
\DeclarePairedDelimiter\abs{\lvert}{\rvert}%
\DeclarePairedDelimiter\norm{\lVert}{\rVert}%
\newcommand{\tr}{\text{tr}}


% Swap the definition of \abs* and \norm*, so that \abs
% and \norm resizes the size of the brackets, and the 
% starred version does not.
\makeatletter
\let\oldabs\abs
\def\abs{\@ifstar{\oldabs}{\oldabs*}}
%
\let\oldnorm\norm
\def\norm{\@ifstar{\oldnorm}{\oldnorm*}}
\makeatother
% % % % % % % % % %
%\providecommand{\abs}[1]{\lvert#1 \rvert}
%\providecommand{\norm}[1]{\lVert#1 \rVert}
%\providecommand{\pin}[2]{\left< #1,#2 \right>} %producto interno

\providecommand{\dpartial}[2]{\frac{\partial #1}{\partial #2}} %derivada parcial

\usepackage{amssymb}
\usepackage{amsmath, amsthm, amsfonts}

%Im�genes
\usepackage{graphicx}
\usepackage{float}
\begin{document}
\fancyhead[L]{Facultad de Ciencias Físicas y Matemáticas}
\fancyhead[R]{Universidad de Chile}

\begin{flushleft}
  \textbf{IN6228 - Teor\'ia de Juegos}
  \\\textbf{Profesor:} Jos\'e Correa.
  \\\textbf{Auxiliares:} Carlos Bonet, Andr\'es Cristi.
\end{flushleft}


\begin{center}
  \Large{\textbf{Tarea 1}}
\end{center}

%\begin{flushleft}



\begin{itemize}
  \item[\textbf{P1.}] Los jugadores 1 y 2 participan en un juego de suma-cero con matrices de pago $A^{1,2}$ y $A^{2,1}$; al mismo tiempo los jugadores 1 y 3 desarrollan otro juego de suma-cero con matrices de pago $A^{1,3}$ y $A^{3,1}$. En vez de ocupar diferentes estrategias en ambos juegos, el jugador 1 debe ocupar la misma estrategia en los dos juegos.
  \begin{enumerate}
    \item Muestre que existe un equilibrio de nash en este juego.
    \item De un algoritmo que permita calcular el equilibrio de nash. Comente sobre la eficiencia de su algoritmo propuesto.
  \end{enumerate}

  \item[\textbf{P2.}] Encuentre todos los equilibrios del juego matricial usando el algoritmo de Lemke-Howson:

     \begin{center}
      \begin{tabular}{|c|c|c|c|}
          \hline
            &L & C & R\\
          \hline\hline
          T&3,4& 5,3 & 2,3\\
          M& 2,5& 3,9& 4,6\\
          B& 3,1 & 2,5& 7,4\\
          \hline
      \end{tabular}
      \end{center}

  \item[\textbf{P3.}] Una jerarquizada manada de $n$ leones encuentran una presa. Si el le\'on 1 decide no comerse a la presa, \'esta escapa y el juego se acaba. Pero si el le\'on 1 se come a la presa, entonces \'el se pone gordo y lento, con esto el le\'on 2 se lo puede comer. Si el le\'on 2 no se come al le\'on 1 entonces el juego termina; sino el le\'on 3 puede comerse al 2 y as\'i sucesivamente se realiza el juego. Suponga que las preferencias de todos los leones son tales que prefieren comer antes de estar sin comida y no comer antes de ser asesinados. Modele el juego de manera extensiva y encuentre todos los EPS del juego.

  \item[\textbf{P4.}] Dos rutas $A$ y $B$ conectan un cierto par 
  origen-destino, con tiempos de viaje que dependen del número de
  usuarios en cada ruta,
  

\end{itemize}

\end{document} 
