\documentclass[11pt, spanish]{article}
\pagestyle{empty}

%acentos de la forma � en vez de \'a
\usepackage[spanish]{babel}
\usepackage[utf8]{inputenc}

%Enumeración con columnas que se usa con \begin{multicols}{2} \begin{enumerate}
\usepackage{multicol}
\usepackage{multirow}


%Para poder usar begin comment
\usepackage{verbatim}

%Teoremas
\usepackage{amsthm}
\theoremstyle{plain}
\newtheorem{teo}{Teorema}


%letras para enumerar
\makeatletter\renewcommand\theenumi{\@alph\c@enumi}\makeatother\renewcommand\labelenumi{\theenumi)}

%m�rgenes
\usepackage[left=2cm,top=2cm,right=2cm]{geometry}


\usepackage{enumerate}

%conjuntos N, Q, R, Z, C
\usepackage{dsfont}
\newcommand{\N}{\mathds{N}}
\newcommand{\Q}{\mathds{Q}}
\newcommand{\R}{\mathds{R}}
\newcommand{\Z}{\mathds{Z}}
\newcommand{\C}{\mathds{C}}
\newcommand{\M}{\mathcal{M}}
\newcommand{\Pol}{\mathcal{P}}

%Transformada de Laplace
\renewcommand{\L}{\mathcal{L}}

%Probabilidades
\newcommand{\PP}{\mathbb{P}}
\newcommand{\E}{\mathbb{E}}
\newcommand{\B}{\mathcal{B}}
\newcommand{\Var}{\mathbb{V}\text{ar}}

%l�gica
\newcommand{\ssi}{\Leftrightarrow}

%funciones
\newcommand{\function}[5]{  \begin{array}{rl}
                                #1: #2 &\longrightarrow #3 \\
                                #4 & \longmapsto #1\left(#4\right)= #5
                            \end{array} }

\newcommand{\funcion}[3]{#1: #2 \longrightarrow #3 }
\newcommand{\parteentera}[1]{[#1]}

%%%operadores matematicos
\usepackage{mathtools}
\DeclarePairedDelimiter\abs{\lvert}{\rvert}%
\DeclarePairedDelimiter\norm{\lVert}{\rVert}%
\newcommand{\tr}{\text{tr}}


% Swap the definition of \abs* and \norm*, so that \abs
% and \norm resizes the size of the brackets, and the 
% starred version does not.
\makeatletter
\let\oldabs\abs
\def\abs{\@ifstar{\oldabs}{\oldabs*}}
%
\let\oldnorm\norm
\def\norm{\@ifstar{\oldnorm}{\oldnorm*}}
\makeatother
% % % % % % % % % %
%\providecommand{\abs}[1]{\lvert#1 \rvert}
%\providecommand{\norm}[1]{\lVert#1 \rVert}
%\providecommand{\pin}[2]{\left< #1,#2 \right>} %producto interno

\providecommand{\dpartial}[2]{\frac{\partial #1}{\partial #2}} %derivada parcial

\usepackage{amssymb}
\usepackage{amsmath, amsthm, amsfonts}


%Grafos

\usepackage{tikz}
\usepackage{float}
\usepackage{graphicx}
\usepackage{verbatim}
\usetikzlibrary{arrows}
\usetikzlibrary{babel}


%Im�genes
\usepackage{graphicx}
\usepackage{float}
\begin{document}

\begin{flushleft}
  \textbf{IN6228-1 Teoría de Juegos 2018}
  \\\textbf{Profesor:} José Correa.
  \\\textbf{Auxiliares:} Carlos Bonet y Andrés Cristi.
\end{flushleft}


\begin{center}
  \large{\textbf{Control 1\\ 27 de Abril de 2018}}
\end{center}

%\begin{flushleft}



\begin{itemize}
\item[\textbf{P1.}] 
\item[\textbf{P2.}] \textbf{Juegos en Redes}

  \begin{enumerate}
    \item Consideremos un juego secuencial entre dos jugadores. En la
      siguiente red ambos jugadores escogen un camino de $s$ a $t$,
      uno despu\'es del otro.
\begin{figure}[H]
\centering
\begin{tikzpicture}[->,shorten >=1pt,auto,node distance=2.5cm,
  thick,main node/.style={circle,fill=blue!20,draw,minimum size=15pt},source node/.style={circle,fill=green!15,draw,minimum size=20pt},dest node/.style={circle,fill=red!20,draw,minimum size=15pt}]
\node[source node] (source) at (-5,0) {$s$};
\node[main node] (sourceB) at (-2,0) {};
\node[main node] (up) at (0.5,1.5) {};;
\node[main node] (down) at (0.5,-1.5) {};
\node[dest node] (sink) at (3,0) {$t$};
\draw (source) to node[below=2] {$1.01x$} (sourceB);
\draw (source) to [bend left=60] node[above=2] {$2.03x$} (sourceB);
\draw (source) to [bend right=80] node[above] {$4x$} (sink);
\draw (sourceB) to node[above=2] {$x$} (up);
\draw (sourceB) to node[above=2] {$0$}  (down);
\draw (up) to node[above=2] {$0$}  (sink);
\draw (up) to node[right] {$0$}  (down);
\draw (down) to node[above=2] {$x$}  (sink);
\end{tikzpicture}
\end{figure}
Calcule el óptimo social, el equilibrio de Nash y el equilibrio perfecto en subjuegos.
Compare los precios de la anarquía de ambos casos.

\item Considere un juego en que $n$ jugadores quieren atravesar de $s$ a $t$
  en un grafo de $m$ arcos paralelos. Cada jugador debe elegir un arco $e$ y
  su costo es $f_e(x)$, con $x$ el número de jugadores que eligen $e$.
  Probaremos que el conjunto de equilibrios perfectos en subjuegos es igual
  al de equilibrios de Nash.
  \begin{figure}[H]
\centering
\begin{tikzpicture}[->,shorten >=1pt,auto,node distance=2.5cm,
  thick,main node/.style={circle,fill=blue!20,draw,minimum size=15pt},source node/.style={circle,fill=green!15,draw,minimum size=20pt},dest node/.style={circle,fill=red!20,draw,minimum size=15pt}]
\node[source node] (source) at (-4,0) {$s$};
\node[dest node] (sink) at (4,0) {$t$};
\node (puntos) at (0,1) {$\vdots$};
\draw (source) to [bend left=80] (sink);
\draw (source) to [bend left=60] (sink);
\draw (source) to [bend left=40] (sink);
\draw (source) to [bend left=10] (sink);
\draw (source) to (sink);
\end{tikzpicture}
\end{figure}
Para ello considere $EQ$ un EPS  y luego
\begin{enumerate}[i.]
  \item Pruebe que, dadas las primeras $n-1$ jugadas, el último jugador siempre juega un equilibrio de Nash.
  \item Suponga que dadas las $k$ primeras jugadas, los últimos $n-k$ jugadores juegan un equilibrio de Nash.
    Sea $e$ el arco que elige el jugador $k$ y suponga que no es equilibrio de Nash, es decir, que existe
    un arco $g$ tal que $ f_e(x_e^{EQ}) > f_g(x_g^{EQ})$. Pruebe que si el jugador $k$ se cambia a $g$
    entonces su costo disminuye.
\end{enumerate}


  \end{enumerate}
\item[\textbf{P3.}] \textbf{Hoteling Game}

  Dos cadenas de comida rápida quieren
  instalar locales en una playa de creciente popularidad y deben decidir en
  qu\'e lugar. Representamos la playa por el intervalo $[0,1]$ y las posiciones
  de los locales como $x,y\in [0,1]$. Los potenciales clientes, que son
  infinitesimalmente pequeños, se distribuyen
  uniformemente en la playa y simplemente eligen la tienda más cercana. Si las
  dos son igualmente cercanas, entonces una mitad va a una y la otra mitad a la
  otra tienda. Cada cadena gana según la masa de clientes que capta.
  Si entonces $x=1/3, y=1$, todos los clientes en $[0,\frac{2}{3})$
  van a la cadena 1, la que tiene una ganancia de $2/3$ y los clientes en
  $(\frac{2}{3},1]$ van a la cadena 2 que tiene una ganancia de $1/3$.

  \begin{enumerate}
    \item Escriba las funciones de utilidad de las cadenas y muestre que no
      son continuas, por lo que no aplican los teoremas de existencia de equilibrio.
    \item Pruebe que $(x,y)= (\frac{1}{2},\frac{1}{2})$ es un EN.
    \item Pruebe que $(\frac{1}{2},\frac{1}{2})$ es el único equilibrio
      en estrategias puras.
    \item Pruebe que $(\frac{1}{2},\frac{1}{2})$ es el único equilibrio en
      estrategias mixtas.
    \item Pruebe que si en vez de dos hay tres cadenas que quieren poner una tienda
      entonces no hay equilibrio en estrategias puras.
  \end{enumerate}

\end{itemize}
\end{document} 
