\documentclass[11pt, spanish]{article}
\pagestyle{empty}

%acentos de la forma � en vez de \'a
\usepackage[spanish]{babel}
\usepackage[utf8]{inputenc}

%Enumeración con columnas que se usa con \begin{multicols}{2} \begin{enumerate}
\usepackage{multicol}
\usepackage{multirow}


%Para poder usar begin comment
\usepackage{verbatim}

%Teoremas
\usepackage{amsthm}
\theoremstyle{plain}
\newtheorem{teo}{Teorema}


%letras para enumerar
\makeatletter\renewcommand\theenumi{\@alph\c@enumi}\makeatother\renewcommand\labelenumi{\theenumi)}

%m�rgenes
\usepackage[left=2cm,top=2cm,right=2cm]{geometry}

\usepackage{fancyhdr}
\pagestyle{fancy}
\usepackage{enumerate}

%conjuntos N, Q, R, Z, C
\usepackage{dsfont}
\newcommand{\N}{\mathds{N}}
\newcommand{\Q}{\mathds{Q}}
\newcommand{\R}{\mathds{R}}
\newcommand{\Z}{\mathds{Z}}
\newcommand{\C}{\mathds{C}}
\newcommand{\M}{\mathcal{M}}
\newcommand{\Pol}{\mathcal{P}}

%Transformada de Laplace
\renewcommand{\L}{\mathcal{L}}

%Probabilidades
\newcommand{\PP}{\mathbb{P}}
\newcommand{\E}{\mathbb{E}}
\newcommand{\B}{\mathcal{B}}
\newcommand{\Var}{\mathbb{V}\text{ar}}

%l�gica
\newcommand{\ssi}{\Leftrightarrow}

%funciones
\newcommand{\function}[5]{  \begin{array}{rl}
                                #1: #2 &\longrightarrow #3 \\
                                #4 & \longmapsto #1\left(#4\right)= #5
                            \end{array} }

\newcommand{\funcion}[3]{#1: #2 \longrightarrow #3 }
\newcommand{\parteentera}[1]{[#1]}

%%%operadores matematicos
\usepackage{mathtools}
\DeclarePairedDelimiter\abs{\lvert}{\rvert}%
\DeclarePairedDelimiter\norm{\lVert}{\rVert}%
\newcommand{\tr}{\text{tr}}


% Swap the definition of \abs* and \norm*, so that \abs
% and \norm resizes the size of the brackets, and the 
% starred version does not.
\makeatletter
\let\oldabs\abs
\def\abs{\@ifstar{\oldabs}{\oldabs*}}
%
\let\oldnorm\norm
\def\norm{\@ifstar{\oldnorm}{\oldnorm*}}
\makeatother
% % % % % % % % % %
%\providecommand{\abs}[1]{\lvert#1 \rvert}
%\providecommand{\norm}[1]{\lVert#1 \rVert}
%\providecommand{\pin}[2]{\left< #1,#2 \right>} %producto interno

\providecommand{\dpartial}[2]{\frac{\partial #1}{\partial #2}} %derivada parcial

\usepackage{amssymb}
\usepackage{amsmath, amsthm, amsfonts}

\newcommand{\vsucc}[0]{\vec{\succ}}


%Im�genes
\usepackage{graphicx}
\usepackage{float}
\begin{document}
\fancyhead[L]{Facultad de Ciencias Físicas y Matemáticas}
\fancyhead[R]{Universidad de Chile}

\begin{flushleft}
  \textbf{IN6228-1 Teoría de Juegos 2018}
  \\\textbf{Profesor:} José Correa.
  \\\textbf{Auxiliares:} Carlos Bonet y Andrés Cristi.
\end{flushleft}


\begin{center}
  \large{\textbf{Clase Auxiliar 8\\ 24 de Agosto de 2018}}
\end{center}

%\begin{flushleft}



\begin{itemize}
\item[\textbf{P1.}] \textbf{Teorema de Gibbard-Satterthwaite}

Una función de elección social $F$ es a prueba de estrategias si $\forall \vsucc, \vsucc', i$ se 
cumple que $F(\succ_i,\succ_{-i}) \succeq_i F(\succ'_i, \succ_{-i})$. En este problema mostraremos el siguiente teorema
\begin{teo}
Si una función de elección social  $F$ sobre al menos tres opciones es sobreyectiva y a prueba de estrategias, entonces es dictatorial.
\end{teo}
\begin{enumerate}
    \item Pruebe que si $F$ es a prueba de estrategias, entonces es monótona (M), i.e. 
    si $F(\vsucc)=a$ y en $\vsucc', a$ está tan bien o mejor que en $\vsucc$
    ($a\succ_i b \Rightarrow a\succ'_i b, \forall i$), entonces
    $F(\vsucc')=a$.
    \item Pruebe que si $F$ es sobreyectiva y monótona, entonces es débilmente pareto
    eficiente (DP): si $\forall i, a\succ_i b$, entonces $F(\vsucc)\neq b$.
    \item Construya la
    función de preferencia social $R$ como sigue: dado $\vsucc$, $F(\vsucc)$ es
    el primero en $R(\vsucc)$; luego se construye $\vsucc^{(2)}$ moviendo a
    $F(\vsucc)$ al final de todas las preferencias y se toma a $F(\vsucc^{(2)})$
    como el segundo en $R(\vsucc)$; y así sucesivamente. Pruebe que si $F$ es (DP) 
    entonces $R$ está bien definida.
    \item Pruebe que si $F$ es (DP) entonces $R$ es unánime.
    \item Pruebe que si $F$ es (M) entonces cumple IIA y concluya.
\end{enumerate}
	
	\item[\textbf{P2.}] \textbf{Bien Público}
	
	En un pueblo el municipio está considerando construir una plaza. El valor
	de construcción es $c>1$. En el pueblo viven $n$ personas, cada una con
	valoración $t_i\in [0,1]$ (estaría dispuestx a pagar $t_i$). Diseñe un
	mecanismo VCG para encontrar la decisión socialmente óptima. Qué pasa
	con los pagos? Es cubierto el costo de la plaza o hay que usar dinero municipal?

\end{itemize}
\end{document} 
