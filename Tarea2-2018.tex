\documentclass[11pt, spanish]{article}
\pagestyle{empty}

%acentos de la forma � en vez de \'a
\usepackage[spanish]{babel}
\usepackage[utf8]{inputenc}

%Enumeración con columnas que se usa con \begin{multicols}{2} \begin{enumerate}
\usepackage{multicol}
\usepackage{multirow}


%Para poder usar begin comment
\usepackage{verbatim}

%Teoremas
\usepackage{amsthm}
\theoremstyle{plain}
\newtheorem{teo}{Teorema}


%letras para enumerar
\makeatletter\renewcommand\theenumi{\@alph\c@enumi}\makeatother\renewcommand\labelenumi{\theenumi)}

%m�rgenes
\usepackage[left=2cm, top=2cm, right=2cm, headheight=13.6pt]{geometry}

\usepackage{fancyhdr}
\pagestyle{fancy}
\usepackage{enumerate}

%conjuntos N, Q, R, Z, C
\usepackage{dsfont}
\newcommand{\N}{\mathds{N}}
\newcommand{\Q}{\mathds{Q}}
\newcommand{\R}{\mathds{R}}
\newcommand{\Z}{\mathds{Z}}
\newcommand{\C}{\mathds{C}}
\newcommand{\M}{\mathcal{M}}
\newcommand{\Pol}{\mathcal{P}}

%Transformada de Laplace
\renewcommand{\L}{\mathcal{L}}

%Probabilidades
\newcommand{\PP}{\mathbb{P}}
\newcommand{\E}{\mathbb{E}}
\newcommand{\B}{\mathcal{B}}
\newcommand{\Var}{\mathbb{V}\text{ar}}

%l�gica
\newcommand{\ssi}{\Leftrightarrow}

%funciones
\newcommand{\function}[5]{  \begin{array}{rl}
                                #1: #2 &\longrightarrow #3 \\
                                #4 & \longmapsto #1\left(#4\right)= #5
                            \end{array} }

\newcommand{\funcion}[3]{#1: #2 \longrightarrow #3 }
\newcommand{\parteentera}[1]{[#1]}

%%%operadores matematicos
\usepackage{mathtools}
\DeclarePairedDelimiter\abs{\lvert}{\rvert}%
\DeclarePairedDelimiter\norm{\lVert}{\rVert}%
\newcommand{\tr}{\text{tr}}


% Swap the definition of \abs* and \norm*, so that \abs
% and \norm resizes the size of the brackets, and the 
% starred version does not.
\makeatletter
\let\oldabs\abs
\def\abs{\@ifstar{\oldabs}{\oldabs*}}
%
\let\oldnorm\norm
\def\norm{\@ifstar{\oldnorm}{\oldnorm*}}
\makeatother
% % % % % % % % % %
%\providecommand{\abs}[1]{\lvert#1 \rvert}
%\providecommand{\norm}[1]{\lVert#1 \rVert}
%\providecommand{\pin}[2]{\left< #1,#2 \right>} %producto interno

\providecommand{\dpartial}[2]{\frac{\partial #1}{\partial #2}} %derivada parcial

\usepackage{amssymb}
\usepackage{amsmath, amsthm, amsfonts}

%Im�genes
\usepackage{graphicx}
\usepackage{float}
\begin{document}
\fancyhead[L]{Facultad de Ciencias Físicas y Matemáticas}
\fancyhead[R]{Universidad de Chile}

\begin{flushleft}
  \textbf{IN6228 - Teor\'ia de Juegos}
  \\\textbf{Profesor:} Jos\'e Correa.
  \\\textbf{Auxiliares:} Carlos Bonet, Andr\'es Cristi.
\end{flushleft}


\begin{center}
  \Large{\textbf{Tarea 2}}
\end{center}

%\begin{flushleft}



\begin{itemize}
  \item[\textbf{P1.}] \textbf{Remate casi óptimo.}

Suponga que debe rematar un ítem (que usted valora
en 0) entre $n$ agentes
con valoraciones no negativas, independientes y regulares pero no idénticamente
distribuidas. Como el remate
de Myerson puede ser bastante complicado en este caso, se le propone
el siguiente mecanismo más simple:
\begin{itemize}
    \item Eliminar a todos los agentes con $c_i(v_i)<0$.
    \item Asignar el ítem al agente con mayor valoración entre los que quedan.
\end{itemize}
donde $c_i(v_i)= v_i - \frac{1-F_i(v_i)}{f_i(v_i)}$ es la valoración virtual del agente $i$.
Compararemos este mecanismo con el remate óptimo (de Myerson).
\begin{enumerate}
    \item Determine los pagos para que el mecanismo propuesto sea compatible en incentivos.
    \item Muestre que si en uno de los mecanismos nadie gana el ítem, entonces en el otro tampoco.
    \item Denotemos por $I$ al ganador en el mecanismo
    óptimo y por $J$ al ganador del propuesto, iguales a $0$ si
    nadie gana.  Pruebe que
    \begin{align}
      \E(c_I(v_I) | I=J)\cdot \PP(I=J) \leq \E(c_J(v_J)) \label{eq1}
    \end{align}
    
    \item Use el hecho de que para todo $i$,
    $c_i(v_i) \leq v_i$ para probar que 
    \begin{align}
      \E( c_I(v_I) | I\neq J) \cdot \PP(I\neq J )\leq \E(\bar{p}_J ) \label{eq2}
    \end{align}
    donde $\bar{p}_i$ es el pago del agente $i$
    bajo el mecanismo propuesto.
    
  \item Use las ecuaciones (\ref{eq1}) y (\ref{eq2}) para concluir que la ganancia esperada bajo el mecanismo
    óptimo es a lo más el doble de la ganancia esperada
    del mecanismo
    propuesto.
\end{enumerate}


  \item[\textbf{P2.}] {\bf Elección Social.}
  
Suponga que los candidatos en una elecci\'on se representan por
elementos de $\R$, i.e.\ los candidatos son un conjunto finito $A
\subseteq \R$. En esta elecci\'on cada votante $i\in N$ tiene un
candidato favorito $p_i\in A$ y prefiere $p'\in A$ por sobre $p''\in
A$ (denotado por $p' \succ_i p''$) si $p''<p'\le p_i$ o bien
$p''>p'\ge p_i$ (si ninguna de estas condiciones se cumple entonces
cualquier preferencia es posible).

\begin{enumerate}
\item Interprete esta condición. ¿Qué situación razonable de elección
modela esto?
\item Suponga que hay un número impar de votantes. Muestre que la
mediana de los $\{p_i\}_{i \in {N}}$ es siempre un ganador de
Condorcet.
\item Considere el sistema que, dados los rankings de todos los
votantes encuentra la mediana de los favoritos declarados de cada uno.
>Es efectivo que, independiente de lo que declaren el resto de los
votantes, a un votante en particular no le conviene mentir en su
ranking?
\item Compare su respuesta con el teorema de Gibbard-Satterthwaite.
\item Considere la siguiente versión del sistema de \emph{approval
voting}. El votante $i\in N$ vota por todos los candidatos $p\in A$
tales que $|{p-p_i}| \le 1$. El sistema luego elige al candidato mas
votado. Muestre un ejemplo en donde el \'unico ganador con este
sistema no resulta ser ganador de Condorcet.
\end{enumerate}

  \item[\textbf{P3.}] {\bf Licitaci\'on de Abastecimiento.} 
  
  Considere una empresa con 2 plantas, $A$ y $B$,
que debe comprar 1 unidad de insumo para cada planta. Existen $n$
proveedores, y el proveedor $i$ tiene un costo marginal de
producci\'on $c_i$. Adem\'as, existe un costo de transporte (conocido)
$d_{ij}$ por llevar una unidad de insumo del  proveedor $i$ a la
planta $j$. \'Este costo lo paga la empresa.


\begin{enumerate}
\item Encuentre el mecanismo con Pivote de Clarke que alcanza la
eficiencia (en este caso, eficiencia corresponde a la minimizaci\'on
de costo total).
\item Ahora consideramos el caso en que la empresa está interesada en minimizar el costo
total esperado a pagar. Para esto, suponga que el costo $c_i$ est\'a
distribuido de acuerdo a $F_i$ en el intervalo
$C=[\underline{c},\bar{c}]$. Un mecanismo estar\'a dado por funciones
de asignaci\'on $x_{iA}:C^n\rightarrow \mathbb{R}$,
$x_{iB}:C^n\rightarrow \mathbb{R}$ y pagos $t:C^n\rightarrow \mathbb{R}^n$.
    \begin{enumerate}

    \item Escriba la funci\'on $U_i(c_i,c_i')$ que representa la
utilidad del proveedor $i$ si su verdadero costo es $c_i$ pero declara un
costo $c_i'$, en t\'erminos de $x_{iA}, x_{iB}$ y $t_i$.

    \item Escriba el problema de maximizaci\'on que debe resolver la empresa

    \item Muestre que $V_i(c_i)= \max_{c_i'\in [\underline{c},\bar{c}]} U_i(c_i,c_i')$ es
      una función convexa y luego pruebe que si un mecanismo es compatible en incentivos
entonces 
\[ R_i(c_i) := E_{c_{-i}}[x_{iA}(c_i,c_{-i})+x_{iB}(c_i,c_{-i})]
\]
es no-creciente en $c_i$ y
$V_i(c_i)=V_i(\bar{c})-\int\limits_{c_i}^{\bar{c}}R_i(s)ds$.

    \item Reescriba la funci\'on objetivo de la empresa de forma que
sea s\'olo funci\'on de $x_{iA}$ y $x_{iB}$.

    \item Encuentre el mecanismo \'optimo (puede asumir propiedades
necesarias sobre $F_i$).

    \end{enumerate}
\end{enumerate}
  
  \item[\textbf{P4.}] {\bf Remate de dos Bienes.} 
  
  Dos bienes indistinguibles serán rematados entre $n$ agentes cuya valoración por el bien sigue una distribución $F$ y son independientes. Cada agente desea obtener solo uno de los bienes, es decir, nunca se asigna dos veces al mismo comprador. En este contexto:
 \begin{enumerate} 
     \item Describa la asignación y los pagos de un mecanismo VCG.
     \item Pruebe que sigue siendo válido el teorema de equivalencia de ingresos.
     \item Asumiendo $F$ regular, encuentre la regla de asignación para un remate de Myerson.
     \item Mecanismo de posted price: formule una expresión para encontrar el mejor precio fijo (el precio que maximiza la utilidad del martillero cuando su estrategia es cobrar el mismo precio a todos los clientes interesados).
     \item Calcule el ingreso del martillero en las partes \textbf{a)}, \textbf{c)} y \textbf{d)} cuando $n=3$ y $F\sim U[0,1]$.
\end{enumerate}
 

\end{itemize}

\end{document} 
