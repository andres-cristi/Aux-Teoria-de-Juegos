\documentclass[11pt, spanish]{article}
\pagestyle{empty}

%acentos de la forma � en vez de \'a
\usepackage[spanish]{babel}
\usepackage[utf8]{inputenc}

%Enumeración con columnas que se usa con \begin{multicols}{2} \begin{enumerate}
\usepackage{multicol}
\usepackage{multirow}


%Para poder usar begin comment
\usepackage{verbatim}

%Teoremas
\usepackage{amsthm}
\theoremstyle{plain}
\newtheorem{teo}{Teorema}


%letras para enumerar
\makeatletter\renewcommand\theenumi{\@alph\c@enumi}\makeatother\renewcommand\labelenumi{\theenumi)}

%m�rgenes
\usepackage[left=2cm,top=2cm,right=2cm]{geometry}

\usepackage{fancyhdr}
\pagestyle{fancy}
\usepackage{enumerate}

%conjuntos N, Q, R, Z, C
\usepackage{dsfont}
\newcommand{\N}{\mathds{N}}
\newcommand{\Q}{\mathds{Q}}
\newcommand{\R}{\mathds{R}}
\newcommand{\Z}{\mathds{Z}}
\newcommand{\C}{\mathds{C}}
\newcommand{\M}{\mathcal{M}}
\newcommand{\Pol}{\mathcal{P}}

%Transformada de Laplace
\renewcommand{\L}{\mathcal{L}}

%Probabilidades
\newcommand{\PP}{\mathbb{P}}
\newcommand{\E}{\mathbb{E}}
\newcommand{\B}{\mathcal{B}}
\newcommand{\Var}{\mathbb{V}\text{ar}}

%l�gica
\newcommand{\ssi}{\Leftrightarrow}

%funciones
\newcommand{\function}[5]{  \begin{array}{rl}
                                #1: #2 &\longrightarrow #3 \\
                                #4 & \longmapsto #1\left(#4\right)= #5
                            \end{array} }

\newcommand{\funcion}[3]{#1: #2 \longrightarrow #3 }
\newcommand{\parteentera}[1]{[#1]}

%%%operadores matematicos
\usepackage{mathtools}
\DeclarePairedDelimiter\abs{\lvert}{\rvert}%
\DeclarePairedDelimiter\norm{\lVert}{\rVert}%
\newcommand{\tr}{\text{tr}}


% Swap the definition of \abs* and \norm*, so that \abs
% and \norm resizes the size of the brackets, and the 
% starred version does not.
\makeatletter
\let\oldabs\abs
\def\abs{\@ifstar{\oldabs}{\oldabs*}}
%
\let\oldnorm\norm
\def\norm{\@ifstar{\oldnorm}{\oldnorm*}}
\makeatother
% % % % % % % % % %
%\providecommand{\abs}[1]{\lvert#1 \rvert}
%\providecommand{\norm}[1]{\lVert#1 \rVert}
%\providecommand{\pin}[2]{\left< #1,#2 \right>} %producto interno

\providecommand{\dpartial}[2]{\frac{\partial #1}{\partial #2}} %derivada parcial

\usepackage{amssymb}
\usepackage{amsmath, amsthm, amsfonts}

%Im�genes
\usepackage{graphicx}
\usepackage{float}
\begin{document}
\fancyhead[L]{Facultad de Ciencias Físicas y Matemáticas}
\fancyhead[R]{Universidad de Chile}

\begin{flushleft}
  \textbf{IN6228-1 Teoría de Juegos 2018}
  \\\textbf{Profesor:} José Correa.
  \\\textbf{Auxiliares:} Carlos Bonet y Andrés Cristi.
\end{flushleft}


\begin{center}
  \large{\textbf{Clase Auxiliar 7\\ 16 de Agosto de 2018}}
\end{center}

%\begin{flushleft}



\begin{itemize}
\item[\textbf{P1.}] \textbf{Myerson.}
\begin{enumerate}
    \item Suponga que quiere subastar un item entre dos personas. Una la valora $v_1\sim$Uniforme$[0,1]$ y la otra
    $v_2\sim$Uniforme$[0,100]$. ¿Cómo sería el remate que maximiza
    la utilidad?
    \item 
\end{enumerate}



\item[\textbf{P2.}] Encuentre el mecanismo óptimo para rematar
un bien entre $n$ agentes con valoraciones i.i.d. en $[\alpha,\beta]$ bajo la restricción de que el bien siempre
tiene que asignarse.

\item[\textbf{P3.}] \textbf{Bulow-Kemperer.}

Suponga que se encuentra en la siguiente situación: quiere
rematar un bien entre $n+1$ agentes con valoraciones i.i.d.
en un intervalo $[\alpha,\beta]$ (usted valora el bien en 0 y
$0<\alpha$), pero uno de los agentes pone como condición para
participar que no haya precio de reserva. ¿Qué será mejor,
aceptar su condición o no?

\item[\textbf{P4.}] \textbf{Remate casi óptimo}

Volvamos al caso en que las distribuciones no son
idénticamente distribuidas (sí regulares). Como el remate
de Myerson es bastante complicado en este caso, considere
el siguiente mecanismo:
\begin{itemize}
    \item Eliminar a todos los agentes con $c_i(v_i)<0$.
    \item Hacer remate de segundo precio entre los que quedan,
    con pago mínimo $c_i^{-1}(0)$.
\end{itemize}
Pruebe que este mecanismo alcanza al menos $1/2$ de la
ganancia óptima.


	

\end{itemize}
\end{document} 
