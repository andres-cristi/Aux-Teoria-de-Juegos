\documentclass[11pt, spanish]{article}
\pagestyle{empty}

%acentos de la forma � en vez de \'a
\usepackage[spanish]{babel}
\usepackage[utf8]{inputenc}

%Enumeración con columnas que se usa con \begin{multicols}{2} \begin{enumerate}
\usepackage{multicol}
\usepackage{multirow}


%Para poder usar begin comment
\usepackage{verbatim}

%Teoremas
\usepackage{amsthm}
\theoremstyle{plain}
\newtheorem{teo}{Teorema}


%letras para enumerar
\makeatletter\renewcommand\theenumi{\@alph\c@enumi}\makeatother\renewcommand\labelenumi{\theenumi)}

%m�rgenes
\usepackage[left=2cm, top=2cm, right=2cm, headheight=13.6pt]{geometry}

\usepackage{fancyhdr}
\pagestyle{fancy}
\usepackage{enumerate}

%conjuntos N, Q, R, Z, C
\usepackage{dsfont}
\newcommand{\N}{\mathds{N}}
\newcommand{\Q}{\mathds{Q}}
\newcommand{\R}{\mathds{R}}
\newcommand{\Z}{\mathds{Z}}
\newcommand{\C}{\mathds{C}}
\newcommand{\M}{\mathcal{M}}
\newcommand{\Pol}{\mathcal{P}}

%Transformada de Laplace
\renewcommand{\L}{\mathcal{L}}

%Probabilidades
\newcommand{\PP}{\mathbb{P}}
\newcommand{\E}{\mathbb{E}}
\newcommand{\B}{\mathcal{B}}
\newcommand{\Var}{\mathbb{V}\text{ar}}

%l�gica
\newcommand{\ssi}{\Leftrightarrow}

%funciones
\newcommand{\function}[5]{  \begin{array}{rl}
                                #1: #2 &\longrightarrow #3 \\
                                #4 & \longmapsto #1\left(#4\right)= #5
                            \end{array} }

\newcommand{\funcion}[3]{#1: #2 \longrightarrow #3 }
\newcommand{\parteentera}[1]{[#1]}

%%%operadores matematicos
\usepackage{mathtools}
\DeclarePairedDelimiter\abs{\lvert}{\rvert}%
\DeclarePairedDelimiter\norm{\lVert}{\rVert}%
\newcommand{\tr}{\text{tr}}


% Swap the definition of \abs* and \norm*, so that \abs
% and \norm resizes the size of the brackets, and the 
% starred version does not.
\makeatletter
\let\oldabs\abs
\def\abs{\@ifstar{\oldabs}{\oldabs*}}
%
\let\oldnorm\norm
\def\norm{\@ifstar{\oldnorm}{\oldnorm*}}
\makeatother
% % % % % % % % % %
%\providecommand{\abs}[1]{\lvert#1 \rvert}
%\providecommand{\norm}[1]{\lVert#1 \rVert}
%\providecommand{\pin}[2]{\left< #1,#2 \right>} %producto interno

\providecommand{\dpartial}[2]{\frac{\partial #1}{\partial #2}} %derivada parcial

\usepackage{amssymb}
\usepackage{amsmath, amsthm, amsfonts}

%Im�genes
\usepackage{graphicx}
\usepackage{float}
\begin{document}
\fancyhead[L]{Facultad de Ciencias Físicas y Matemáticas}
\fancyhead[R]{Universidad de Chile}

\begin{flushleft}
  \textbf{IN6228 - Teor\'ia de Juegos}
  \\\textbf{Profesor:} Jos\'e Correa.
  \\\textbf{Auxiliares:} Carlos Bonet, Andr\'es Cristi.
\end{flushleft}


\begin{center}
  \Large{\textbf{Tarea 2}}
\end{center}

%\begin{flushleft}



\begin{itemize}
  \item[\textbf{P1.}] \textbf{Remate casi óptimo}

Suponga que debe rematar un ítem (que usted valora
en 0) entre $n$ agentes
con valoraciones independientes pero no idénticamente
distribuidas. Como el remate
de Myerson puede ser bastante complicado en este caso, considere
el siguiente mecanismo:
\begin{itemize}
    \item Eliminar a todos los agentes con $c_i(v_i)<0$.
    \item Asignar el ítem al agente con mayor valoración entre los que quedan.
\end{itemize}
donde $c_i(v_i)= v_i - \frac{1-F_i(v_i)}{f_i(v_i)}$ es la valoración virtual del agente $i$.
\begin{enumerate}
    \item Determine los pagos para que el mecanismo sea compatible en incentivos.
    \item Denotemos por $I$ al ganador en el mecanismo
    óptimo y por $J$ al ganador del propuesto, iguales a $0$ si
    nadie gana. Considere además el evento 
    $A= \{ I\neq 0, J\neq 0\}$. Pruebe que
    \begin{align}
        \E(c_I(v_I) | I=J, A) \leq \E(c_J(v_J)|A).
    \end{align}
    
    \item Use el hecho de que para todo $i$,
    $c_i(v_i) \leq v_i$ para probar que 
    \begin{align}
        \E( c_I(v_I) | I\neq J, A) \leq \E(\bar{p}_J | A)
    \end{align}
    donde $\bar{p}_i$ es el pago del agente $i$
    bajo el mecanismo propuesto.
    \item Muestre que, condicional en $A^c$, ambos
    mecanismos reportan una ganancia igual a cero.
    
    \item Use los tres puntos anteriores para concluir que la ganancia esperada bajo el mecanismo
    óptimo es a lo más el doble de la ganancia esperada
    del mecanismo
    propuesto.
\end{enumerate}


  \item[\textbf{P2.}] 

  \item[\textbf{P3.}] 
  
  \item[\textbf{P4.}] 
 

\end{itemize}

\end{document} 
